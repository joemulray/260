%
% compile with pdflatex:
%   $ pdflatex homework-example.tex
% , yields homework-example.pdf
%    You might need to compile twice, if, e.g., you start using references
%

\documentclass[10pt,letterpaper,oneside]{article}
\usepackage[ascii]{inputenc}
\usepackage{amsmath,amsfonts,amssymb}
\usepackage[margin=1in]{geometry}
	\setlength{\parindent}{0em}
	\setlength{\parskip}{1em}

\newtheorem{theorem}{Theorem}

\usepackage[section]{algorithm}
%\usepackage{algorithm2e}
\usepackage[noend]{algpseudocode}

	%\renewcommand{\algorithmiccomment}[1]%
	% {\bgroup\hfill\small$\triangleright$#1\egroup}

	%\renewcommand{\algorithmiccomment}[1]{\footnotesize // #1}
	\renewcommand{\algorithmicforall}{$\boldsymbol\forall$}

%%%% user definitions %%%%%%%%%%%%%%%%%%%%%%%

\newcommand{\Problem}[1]{\subsection*{Problem #1}}
\newcommand{\Part}[1]{\subsubsection*{Part #1}}
\newcommand{\Solution}{\subsubsection*{Solution}}

	% Forms for Big-Oh notation
\DeclareMathOperator{\Omicron}{O}
\DeclareMathOperator{\omicron}{o}

\newcommand{\BigOh}[1]{\Omicron(#1)}
\newcommand{\LittleOh}[1]{\omicron(#1)}
\newcommand{\BigOmega}[1]{\Omega(#1)}
\newcommand{\LittleOmega}[1]{\omega(#1)}
\newcommand{\BigTheta}[1]{\Theta(#1)}

	% Operators for dominance notation
\newcommand{\domeq}{\sim}
\newcommand{\domle}{\preceq}
\newcommand{\domlt}{\prec}
\newcommand{\domge}{\succeq}
\newcommand{\domgt}{\succ}

	% Some convenient definitions for coding
\newcommand{\NOT}{~\textbf{not}~}
\newcommand{\AND}{~\textbf{and}~}
\newcommand{\OR}{~\textbf{or}~}
\newcommand{\TRUE}{\textsc{True}}
\newcommand{\FALSE}{\textsc{False}}

%%%%%%%%  You edit stuff below this line  %%%%%%%%%%%%%%%%%%%%%%%%%%

\title{Homework 1	\LaTeX}
\author{Joseph Mulray}
%\date{16 January 2017}  % uses today, by default

\begin{document}

\maketitle

\Problem{1.10}

	Indicate for each distinct pair i and j whether fi(n) is O(fj(n)) and whether fi(n) is Ω(fj(n)).
	\Part{a}
	\Solution
	$f_1 = n^2$\\
	$f_2 = n^2 + 1000n$
	\[
		f_3= \left\lbrace%
		\begin{array}{cc}
		n, \text{$n$ is odd}\\
		n^3, \text{$n$ is even}\\
		\end{array} \right.
	\]
	\[
		f_4= \left\lbrace%
		\begin{array}{cc}
		n, \text{$n$ is n <= 100}\\
		n^3, \text{$n$ is even}\\
		\end{array} \right.
	\]	

	\begin{eqnarray*}
		\text{Let $n = 5$.  Substituting:} \\
		f(5) = 3(5)^2 + 2(5) - 17 \\
		f(5) = 3*25 + 10 - 17 \\
		\text{So, of course:} \\
		f(5) = 68 
	\end{eqnarray*}

	We can get the equals signs to line up:

	\begin{eqnarray*}
		\text{Let $n = 5$.  Substituting:} \\
		f(5) & = & 3(5)^2 + 2(5) - 17 \\
		     & = & 3*25 + 10 - 17 \\
		\text{And, here's text:} \\
		f(5) & = & 68 
	\end{eqnarray*}

\Problem{2}

	\Part{a}
	Here are some sums you'd better have stuck in your head

	\Solution
		\begin{eqnarray}
			\sum_{i=a}^b r  & = & (b-a+1)r \\
			\sum_{i} c(f_i) & = & c\sum_{i} (f_i) \\
			\text{For some big parens:} \\
			\sum_{i} (f_i+g_i) & = & \left(\sum_{i} f_i + \sum_{i} g_i\right) \\
			\sum_{i=1}^m i  & = & \frac{m(m+1)}{2} \\
			\sum_{i=1}^m i^2  & = & \frac{m(m+1)(2m+1)}{6} \\
			\sum_{i=0}^m ar^i, r\neq1  & = & a\frac{r^{m+1}-1}{r-1} \\
			\sum_{i=0}^\infty ar^i, 0<|r|<1  & = & a\frac{1}{1-r} 
		\end{eqnarray}
		
	\Part{b}
	Here are some logs you'd better have stuck in your head

	\Solution
		\begin{eqnarray}
			\log_b 1 & = & 0 \\
			\log_b b & = & 1 \\
			\log_b (xy) & = & \log_b x + \log_b y \\
			\log_b\frac{x}{y} & = & \log_b x - \log_b y \\
			\log_b x^n & = & n\log_b x 
		\end{eqnarray}

\Problem{3}
	Here's a definition of Fibonacci numbers

	\Solution
		\[
			F_n = \left\lbrace%I
			\begin{array}{cc}
				F_{n-1} + F_{n-2} &, n > 1 \\
				1 &, n=0 \\
				1 &, n=1 \\
			\end{array} \right.
		\]

\Problem{4}
	And tables are pretty easy.  \& separates columns, and
	\textbackslash{}\textbackslash{} is a newline in \LaTeX

	\Solution
	\begin{center}
	\begin{tabular}{ c | r | l }
		Name & $n$ & $(3/2)^n$ \\
		\hline
		Picard & 0  & 1 \\
		Riker & 1  & 1.5 \\
		Worf & 2	& 2.25 \\
		Troi & 3	& 3.375 \\
		Crusher & 4	& 5.0625 \\
		LaForge & 5	& 7.59375 \\
		O'Brien & 6	& 11.390625 \\
		Guinan & 7	& 17.0859375 \\
		Q & 8	& 25.62890625 
	\end{tabular}
	\end{center}

	We can get the decimals to line up:

	\begin{center}
	\begin{tabular}{ c | r | r @{.} l }
		Name & $n$ & $(3/2)^n$ \\
		\hline
		Picard & 0  & 1 & \\
		Riker & 1  & 1 & 5 \\
		Worf & 2	& 2 & 25 \\
		Troi & 3	& 3 & 375 \\
		Crusher & 4	& 5 & 0625 \\
		LaForge & 5	& 7 & 59375 \\
		O'Brien & 6	& 11 & 390625 \\
		Guinan & 7	& 17 & 0859375 \\
		Q & 8	& 25 & 62890625 
	\end{tabular}
	\end{center}

\Problem{5}
Show the following statements:

	\Part{a}
		$ 3n+7 \in \BigOh{n^2} $
	
	\Solution
		By the definition, we need to find a $c>0$ and $n_0>0$ such that $cn^2
		\geq 3n+7 , \forall n>n_0$:

		\begin{eqnarray*}
			3n+7 & \leq & 3n^2 + 7n^2 , \forall n\geq1 \\
			     & =    & 10n^2 \\
			\text{So, we have} \\
			10n^2 & \geq & 3n+7 , \forall n>1
		\end{eqnarray*}

		We have our 2 witnesses.


\Problem{6}
Write some nonsense code, by way of example.

	\Solution
	Here's a list of common math symbols in $LaTeX$:  https://oeis.org/wiki/List\_of\_LaTeX\_mathematical\_symbols

	\begin{algorithm}[H]  % Optional float wrapper for algorithmic, algorithmicx
		\caption{Totally optional}

		\begin{algorithmic} % [1] if followed by an arg, line numbering starts there
			\Function{Foo}{$G$:~graph, $c$:~color}
				\State $S \gets$ \textsc{Set}()
				\State $rv \gets 13$
				\State \Comment{This is a comment -- $\BigOh{V^2}$}
				\ForAll{$v \in G$}
					\State $color[v] \gets white$
				\EndFor

				\State \Call{bar}{$Q, c$}

				\item[] % blank line, not numbered
				\While{ \NOT \Call{IsEmpty}{$S$} \AND \Call{Magic}{$v$} }
					\If{ $x \leq \infty$ }
						\State $\pi[w] \gets v$
						\State \Call{Insert}{ $S, v$ }
					\Else
						\State $\pi[w] \gets u$
					\EndIf
					\State \textsc{Juggle\_Magic}()
				\EndWhile
				\item[]
				\Return $S$
			\EndFunction

			\State  % blank line, numbered with its neighbors
			\For{ $i \gets 1..n$ }
				\For{ $j \gets 1..i$ }
					\State \Call{Print}{ i, j }
				\EndFor
			\EndFor
		\end{algorithmic}
	\end{algorithm}
 

\end{document}
